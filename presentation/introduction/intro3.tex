{\color{blue}{\footnotesize Alice is given input $x$.}}
\pause
\begin{leftbubbles}
{\footnotesize Hello Bob.  I can't reveal $x$, but $p_1(x)$ is \texttt{p1}.}
\end{leftbubbles}
\pause
\rightline{{\color{olive}{\footnotesize Bob is given input $y$.}}}
\pause
\begin{rightbubbles}
{\footnotesize Thanks Alice.  I can't reveal $y$, but $p_2(y, \texttt{p1})$ is \texttt{p2}.}
\end{rightbubbles}
\pause
\center{\footnotesize{... yada yada yada ...}}
\pause
\begin{leftbubbles}
{\footnotesize Pleasure doing business with you Bob.  My final clue for you is that $p_{n-1}(x, \texttt{p1}, ..., \texttt{pn-2})$ is \texttt{pn-1}.}
\end{leftbubbles}
\pause
\begin{rightbubbles}
{\footnotesize Rad.  Then $p_n(y, \texttt{p1}, ..., \texttt{pn-1})$ is \texttt{pn}.  TTFN!}
\end{rightbubbles}