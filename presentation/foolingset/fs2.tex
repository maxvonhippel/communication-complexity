\begin{definition}[Fooling Set]
If $f : \mathbb{B}^n \times \mathbb{B}^n \to \mathbb{B}$ is a function,
	a \emph{fooling set} for $f$ is a set
	\(S \subseteq \mathbb{B}^n \times \mathbb{B}^n\)
	such that for some choice $b \in \mathbb{B}$
\(f(S) = \{ b \}\)
but, for all distinct $(x, y), (x', y') \in S$,
either $f(x, y') \neq b \text{ or } f(x', y) \neq b$
\end{definition}
Basically, a fooling set is a group of inputs that go to the same output,
but which is \emph{brittle} to argument-swapping.
In some sense these \emph{brittle} sets lower-bound the difficulty in grouping like inputs.
\pause
\begin{lemma}[Fooling Set Method]
If $f$ has a size-$M$ fooling set, then $C(f) \geq \text{log}_2(M)$.
\end{lemma}