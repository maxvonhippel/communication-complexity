With the \emph{fooling set} method, we lower-bounded $C(f)$.
Now we'll introduce a new method that both lower- and upper-bounds $C(f)$.
\pause
\begin{definition}[$M(f)$]
The \emph{matrix of }$f$,
	denoted $M(f)$,
		is the $2^n \times 2^n$ matrix whose $(x, y)$th entry
		is the value $f(x, y)$.
\end{definition}
\pause
\begin{example}[$M(\lor)$]
\adjustbox{max width=.25\textwidth}{
	\begin{tabular}{llllll}
	   & {\cellcolor{blue!20}{00}} & 
	     {\cellcolor{blue!20}{01}} & 
	     {\cellcolor{blue!20}{10}} & 
	     {\cellcolor{blue!20}{11}} \\
	{\cellcolor{green!20}{00}} & 00 & 01 & 10 & 11 \\
	{\cellcolor{green!20}{01}} & 01 & 01 & 11 & 11 \\
	{\cellcolor{green!20}{10}} & 10 & 11 & 10 & 11 \\
	{\cellcolor{green!20}{11}} & 11 & 11 & 11 & 11
	\end{tabular}
}
\adjustbox{max width=.73\textwidth}{
\begin{minipage}{\linewidth}
\begin{itemize}
	\item The green cells are Alice's possible inputs $x$.
	\item The blue cells are Bob's possible inputs $y$.
	\item The uncolored cells are the matrix $M(f)$.
\end{itemize}
\end{minipage}
}
\end{example}