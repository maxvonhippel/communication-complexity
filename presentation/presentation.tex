\documentclass{beamer}
\usetheme{Berkeley}

\include{presentation-header}


\title{Communication Complexity}
\author{Jake Kinsella and Max von Hippel}
\institute{Northeastern University}
\date{\today}



\begin{document}

\frame{\titlepage}

\begin{frame}
\frametitle{Communication Complexity}
\emph{If Alice knows $x$, and Bob knows $y$, how many bits of information must they communicate, in order for both Alice and Bob to know $f(x, y)$?}
\end{frame}

\begin{frame}
\tableofcontents
\end{frame}

% 0  - 5m   - Max introduces the problem
% 5  - 10m  - Jake a couple examples
% 10 - 30m  - Jake does fooling set
% 30 - 50m  - Max does the tiling method
% 50 - 60m  - Max does 2-party discrepancy method
% 60 - 65m  - Jake introduces motivating problem for multi-party version
% 65 - 70m  - Jake defines generalizes multi-party version of problem 
% 70 - 85m  - Max gives multi-party discrepancy without proof
% 85 - 90m  - Jake does non-deterministic version
% 90 - 95m  - Max does randomized

\section{Introduction}

\begin{frame}{(Max)}
Consider a two-party communication problem,
	in which the participants

\begin{figure}[h]
\centering
\begin{subfigure}{.3\textwidth}
  \centering
  \includegraphics[width=.2\linewidth]{introduction/alice.png}
  \caption{Alice}
  \label{fig:alice}
\end{subfigure}%
and
\begin{subfigure}{.3\textwidth}
  \centering
  \includegraphics[width=.2\linewidth]{introduction/bob.png}
  \caption{Bob}
  \label{fig:bob}
\end{subfigure}
\label{fig:participants}
\end{figure}

\emph{participate} to compute a function:
\[f : \underbrace{\mathbb{B}^n}_{\begin{aligned}\text{Alice's}\\\text{input}\end{aligned}} \times 
      \underbrace{\mathbb{B}^n}_{\begin{aligned}\text{Bob's}\\\text{input}\end{aligned}} \to 
      \underbrace{\mathbb{B}}_{\begin{aligned}\text{global}\\\text{output}\end{aligned}}\]
\end{frame}

\begin{frame}{(Max)}
The players can come up with a \emph{protocol} 
\(\Pi = (p_1, ..., p_t)\),
namely, 
	for some natural $t \in \mathbb{N}$,
	a sequence of $t$-many functions
\(p_i : \mathbb{B}^* \to \mathbb{B}^*\)
such that the communication between the players looks like this ...
\end{frame}

\begin{frame}{(Max)}
{\color{blue}{\footnotesize Alice is given input $x$.}}
\pause
\begin{leftbubbles}
{\footnotesize Hello Bob.  I can't reveal $x$, but $p_1(x)$ is \texttt{p1}.}
\end{leftbubbles}
\pause
\rightline{{\color{olive}{\footnotesize Bob is given input $y$.}}}
\pause
\begin{rightbubbles}
{\footnotesize Thanks Alice.  I can't reveal $y$, but $p_2(y, \texttt{p1})$ is \texttt{p2}.}
\end{rightbubbles}
\pause
\center{\footnotesize{... yada yada yada ...}}
\pause
\begin{leftbubbles}
{\footnotesize Pleasure doing business with you Bob.  My final clue for you is that $p_{n-1}(x, \texttt{p1}, ..., \texttt{pn-2})$ is \texttt{pn-1}.}
\end{leftbubbles}
\pause
\begin{rightbubbles}
{\footnotesize Rad.  Then $p_n(y, \texttt{p1}, ..., \texttt{pn-1})$ is \texttt{pn}.  TTFN!}
\end{rightbubbles}
\end{frame}

\begin{frame}{(Max)}
\begin{itemize}
\item The functions $p_i$ can be \emph{anything} so long as they are well-defined.  E.g., could solve the Halting Problem.
\item After the final message, \emph{both parties} must know $f(x, y)$.
\end{itemize}
\pause
\begin{definition}[Communication Complexity]
Suppose $\Pi$ is a protocol for $f$ in which at most $t$
bits are communicated between Alice and Bob.
Then the \emph{communication complexity} of $\Pi$,
	denoted $C(\Pi)$, is $t$.
\end{definition}
\pause
\begin{definition}[$C(f)$]
The communication complexity of $f$, denoted $C(f)$,
	is the minimum communication complexity
	achieved by any protocol for $f$.
\end{definition}
\end{frame}

\subsection{Examples}

\begin{frame}{Parity (Jake)}
\begin{example}[Are the number of \texttt{1}s in $xy$ even (0), or odd (1)?]
\(f : \mathbb{B}^n \times \mathbb{B}^n \to \mathbb{B}\)
is precisely
\((x, y) \mapsto \bigoplus xy\).
\end{example}
\pause
Example protocol $\Pi$:
\begin{leftbubbles}
{\footnotesize $\texttt{P1}=\text{parity}(x)$.}
\end{leftbubbles}
\pause
\begin{rightbubbles}
{\footnotesize $\texttt{P2}=\text{parity}(y) \bigoplus \texttt{P1}$}
\end{rightbubbles}
\pause
Now both Alice and Bob know $f(x, y) = \texttt{P2}$.
$C(f) \leq 2$ because $C(\Pi) = 2$ and $\Pi$ implements $f$.
But $C(f) \geq 2$ because $f$ depends on $x$ and $y$.
Hence \(C(f) = 2\).
\end{frame}

\begin{frame}{Halting (Jake)}
\textbf{Example 2: Halting}
\\

\par{Function $H: \{0, 1\}^{n} \times \{0, 1\}^{n} \rightarrow \{0, 1\}$}
\\

\par{$x = 1^{n}$ and $y=<M>$}
\\

\par{H returns 1 if M halts on x}
\\

\textbf{Protocol $\Pi$}
\begin{center}
  \begin{tabular}{ |m{15em}|m{15em}| } 
    \hline
    Player 1                                  & Player 2 \\ [0.5ex] 
    \hline
    $x=1^{10}$                                & $y=<M_{accept}>$ \\
    $P_{1}(1^{10})= 1 \longrightarrow$        &  \\
                                              & $p_{1} = 1$ \\
                                              & $P_{2}(y, p_{1}) = M_{accept}(1^{|<M_{accept}>|})$ \\
                                              & $\longleftarrow M(1^{10}) = 1$ \\
    $p_{2} = 1$                               & \\
    \hline
  \end{tabular}
\end{center}

\par{In communication complexity problems, both players have unlimited computation power. This allows Player 2 to solve
the halting problem. Computational power and time is ignored to focus on communication between players.}
\end{frame}

\section{Methods}

\subsection{2-Party Problem}

\begin{frame}{Methods (Max)}
If we find a protocol $\Pi$,
	then we know $C(f)$ is at most $C(\Pi)$.
\pause

What if we don't know any protocol $\Pi$?
\pause
\begin{itemize}
	\item Could we upper-bound $C(f)$ without knowing $\Pi$?
\end{itemize}

\pause
What if the only protocols we find seem really lousy?
\pause
\begin{itemize}
	\item Could we lower-bound $C(f)$ without finding a better protocol?
\end{itemize}

\pause
\centering{\textbf{TL;DR: yup.}}
\end{frame}

\subsubsection{Fooling Set Method}

\begin{frame}{Fooling Set Method (Jake)}
We begin with a motivating observation.
\pause
\begin{lemma}[Communication Equality is Image Equality]
If Alice and Bob exchange the same sequence of messages when Alice gets $\textbf{x}$ and Bob gets $\textbf{y}$ as they do when Alice gets $\textbf{x}'$ and Bob gets $\textbf{y}'$, then $f(\textbf{x}, \textbf{y}) = f(\textbf{x}', \textbf{y}')$.
\end{lemma}
\pause
\begin{proof}
$\Pi$ is deterministic and $f$ is a function.
\end{proof}
\pause
\underline{Idea:} an efficient protocol will efficiently group together inputs that go to the same output.
\end{frame}

\begin{frame}{Fooling Set Method (Jake)}
\underline{Idea:} an efficient protocol will efficiently \textbf{group together inputs that go to the same output.}

\begin{definition}[Fooling Set]
If $f : \mathbb{B}^n \times \mathbb{B}^n \to \mathbb{B}$ is a function,
	a \emph{fooling set} for $f$ is a set
	\(S \subseteq \mathbb{B}^n \times \mathbb{B}^n\)
	such that for some choice $b \in \mathbb{B}$
\(f(S) = \{ b \}\)
but, for all distinct $(x, y), (x', y') \in S$,
\((\neg b) \in f(\{ x, x' \} \times \{ y, y' \})\).
\end{definition}
Basically, a fooling set is a group of inputs that go to the same output,
but which is \emph{brittle} to argument-swapping.
In some sense these \emph{brittle} sets lower-bound the difficulty in grouping like inputs.
\pause
\begin{lemma}[Fooling Set Method]
If $f$ has a size-$M$ fooling set, then $C(f) \geq \text{log}_2(M)$.
\end{lemma}
\end{frame}

\begin{frame}{Fooling Set Method (Jake)}
\begin{example}[Set-Disjointness]
$\textsc{Disj} : \mathbb{B}^n \times \mathbb{B}^n \to \mathbb{B}$ is the function
that maps $(A, B)$ to 1 if $A \cap B = \emptyset$ else 0.
\end{example}
How many fooling sets does \textsc{Disj} have?
\pause
Notice $A, B$ are disjoint iff $A \oplus B$ is \textbf{1}.
\pause
There are $2^n$ possible values $A$.
\pause
Hence $2^n$ values $(A, B)$ s.t. $A \oplus B = \textbf{1}$.
None of these distinct $(A, B), (A', B')$ satisfy
 $A \oplus B = A \oplus B'$ or
$A \oplus B = A' \oplus B$
else they wouldn't be distinct.
\pause
So we get a $2^n$-size fooling set.
\pause
\[\therefore \, C(\textsc{Disj}) \geq \text{log}_2(2^n) = n\]
\end{frame}

\begin{frame}{Fooling Set Method (Jake)}
\textbf{NTS:} If $f$ has a size-$M$ fooling set then $C(f) \geq \text{log}_2(M)$.
\begin{proof}
For a contradiction suppose a protocol $\Pi$ exists for $f$ s.t. $C(\Pi) < \text{log}_2(M)$.
\pause
Then $\Pi$ yields at most $2^{C(\Pi)} < 2^{\text{log}_2(M)} = M$
distinct communication patterns.
\pause
However, there are $M$ input pairs $(x, y)$ that each have distinct communication patterns by definition.
\pause
Since $2^{C(\Pi)} < M$ there must be some $(x, y), (x', y')$ on which $\Pi$ yields the same communication pattern.

\medskip

\pause
Then $(x, y')$ must yield the same communication pattern as $(x, y)$
as Bob cannot possibly tell the difference.  The argument is symmetric for $(x', y)$ and Alice.  One of the two must yield a contradiction and we are done.
\end{proof}
\end{frame}

\subsubsection{Tiling Method}

\begin{frame}{Tiling Method (Max)}
\TODO
\end{frame}

\subsubsection{Discrepency Method}

\begin{frame}{2-Party Discrepency Method (Max)}
\TODO
\end{frame}

\subsection{Multi-Party Problem}

\begin{frame}{Multi-Party Problem (Jake)}
\TODO
\end{frame}

\subsubsection{Multi-Party Discrepency Method}

\begin{frame}{Multi-Party Discrepency Method (Max)}
\TODO
\end{frame}

\section{Other Variants}

\subsection{Non-Deterministic}

\begin{frame}{Non-Deterministic (Jake)}
\TODO
\end{frame}

\subsection{Randomized}

\begin{frame}{Randomized (Max)}
\TODO
\end{frame}

\end{document}