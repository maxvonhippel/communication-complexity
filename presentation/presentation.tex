\documentclass{beamer}
\usetheme{Berkeley}

\include{presentation-header}


\title{Communication Complexity}
\author{Jake Kinsella and Max von Hippel}
\institute{Northeastern University}
\date{\today}



\begin{document}

\frame{\titlepage}

\begin{frame}
\frametitle{Communication Complexity}
\emph{If Alice knows $x$, and Bob knows $y$, how many bits of information must they communicate, in order for both Alice and Bob to know $f(x, y)$?}
\end{frame}

\begin{frame}
\tableofcontents
\end{frame}

% 0  - 5m   - Max introduces the problem
% 5  - 10m  - Jake a couple examples
% 10 - 30m  - Jake does fooling set
% 30 - 50m  - Max does the tiling method
% 50 - 60m  - Max does 2-party discrepancy method
% 60 - 65m  - Jake introduces motivating problem for multi-party version
% 65 - 70m  - Jake defines generalizes multi-party version of problem 
% 70 - 85m  - Max gives multi-party discrepancy without proof
% 85 - 90m  - Jake does non-deterministic version
% 90 - 95m  - Max does randomized

\section{Introduction}

\begin{frame}{Introduction (Max)}
Consider a two-party communication problem,
	in which the participants

\begin{figure}[h]
\centering
\begin{subfigure}{.3\textwidth}
  \centering
  \includegraphics[width=.2\linewidth]{introduction/alice.png}
  \caption{Alice}
  \label{fig:alice}
\end{subfigure}%
and
\begin{subfigure}{.3\textwidth}
  \centering
  \includegraphics[width=.2\linewidth]{introduction/bob.png}
  \caption{Bob}
  \label{fig:bob}
\end{subfigure}
\label{fig:participants}
\end{figure}

\emph{participate} to compute a function:
\[f : \underbrace{\mathbb{B}^n}_{\begin{aligned}\text{Alice's}\\\text{input}\end{aligned}} \times 
      \underbrace{\mathbb{B}^n}_{\begin{aligned}\text{Bob's}\\\text{input}\end{aligned}} \to 
      \underbrace{\mathbb{B}}_{\begin{aligned}\text{global}\\\text{output}\end{aligned}}\]
\end{frame}

\begin{frame}{Introduction (Max)}
The players can come up with a \emph{protocol} 
\(\Pi = (p_1, ..., p_t)\),
namely, 
	for some natural $t \in \mathbb{N}$,
	a sequence of $t$-many functions
\(p_i : \mathbb{B}^* \to \mathbb{B}^*\)
such that the communication between the players looks like this ...
\end{frame}

\begin{frame}{Introduction (Max)}
{\color{blue}{\footnotesize Alice is given input $x$.}}
\pause
\begin{leftbubbles}
{\footnotesize Hello Bob.  I can't reveal $x$, but $p_1(x)$ is \texttt{p1}.}
\end{leftbubbles}
\pause
\rightline{{\color{olive}{\footnotesize Bob is given input $y$.}}}
\pause
\begin{rightbubbles}
{\footnotesize Thanks Alice.  I can't reveal $y$, but $p_2(y, \texttt{p1})$ is \texttt{p2}.}
\end{rightbubbles}
\pause
\center{\footnotesize{... yada yada yada ...}}
\pause
\begin{leftbubbles}
{\footnotesize Pleasure doing business with you Bob.  My final clue for you is that $p_{n-1}(x, \texttt{p1}, ..., \texttt{pn-2})$ is \texttt{pn-1}.}
\end{leftbubbles}
\pause
\begin{rightbubbles}
{\footnotesize Rad.  Then $p_n(y, \texttt{p1}, ..., \texttt{pn-1})$ is \texttt{pn}.  TTFN!}
\end{rightbubbles}
\end{frame}

\begin{frame}{Introduction (Max)}
\begin{itemize}
\item The functions $p_i$ can be \emph{anything} so long as they are well-defined.  E.g., could solve the Halting Problem.
\item After the final message, \emph{both parties} must know $f(x, y)$.
\end{itemize}
\pause
\begin{definition}[Communication Complexity]
Suppose $\Pi$ is a protocol for $f$ in which at most $t$
bits are communicated between Alice and Bob.
Then the \emph{communication complexity} of $\Pi$,
	denoted $C(\Pi)$, is $t$.
\end{definition}
\pause
\begin{definition}[$C(f)$]
The communication complexity of $f$, denoted $C(f)$,
	is the minimum communication complexity
	achieved by any protocol for $f$.
\end{definition}
\end{frame}

\subsection{Examples}

\begin{frame}{Parity (Jake)}
\begin{example}[Are the number of \texttt{1}s in $xy$ even (0), or odd (1)?]
\(f : \mathbb{B}^n \times \mathbb{B}^n \to \mathbb{B}\)
is precisely
\((x, y) \mapsto \bigoplus xy\).
\end{example}
\pause
Example protocol $\Pi$:
\begin{leftbubbles}
{\footnotesize $\texttt{P1}=\text{parity}(x)$.}
\end{leftbubbles}
\pause
\begin{rightbubbles}
{\footnotesize $\texttt{P2}=\text{parity}(y) \bigoplus \texttt{P1}$}
\end{rightbubbles}
\pause
Now both Alice and Bob know $f(x, y) = \texttt{P2}$.
$C(f) \leq 2$ because $C(\Pi) = 2$ and $\Pi$ implements $f$.
But $C(f) \geq 2$ because $f$ depends on $x$ and $y$.
Hence \(C(f) = 2\).
\end{frame}

\begin{frame}{Halting (Jake)}
\textbf{Example 2: Halting}
\\

\par{Function $H: \{0, 1\}^{n} \times \{0, 1\}^{n} \rightarrow \{0, 1\}$}
\\

\par{$x = 1^{n}$ and $y=<M>$}
\\

\par{H returns 1 if M halts on x}
\\

\textbf{Protocol $\Pi$}
\begin{center}
  \begin{tabular}{ |m{15em}|m{15em}| } 
    \hline
    Player 1                                  & Player 2 \\ [0.5ex] 
    \hline
    $x=1^{10}$                                & $y=<M_{accept}>$ \\
    $P_{1}(1^{10})= 1 \longrightarrow$        &  \\
                                              & $p_{1} = 1$ \\
                                              & $P_{2}(y, p_{1}) = M_{accept}(1^{|<M_{accept}>|})$ \\
                                              & $\longleftarrow M(1^{10}) = 1$ \\
    $p_{2} = 1$                               & \\
    \hline
  \end{tabular}
\end{center}

\par{In communication complexity problems, both players have unlimited computation power. This allows Player 2 to solve
the halting problem. Computational power and time is ignored to focus on communication between players.}
\end{frame}

\section{Methods}

\subsection{2-Party Problem}

\begin{frame}{Methods (Max)}
If we find a protocol $\Pi$,
	then we know $C(f)$ is at most $C(\Pi)$.
\pause

What if we don't know any protocol $\Pi$?
\pause
\begin{itemize}
	\item Could we upper-bound $C(f)$ without knowing $\Pi$?
\end{itemize}

\pause
What if the only protocols we find seem really lousy?
\pause
\begin{itemize}
	\item Could we lower-bound $C(f)$ without finding a better protocol?
\end{itemize}

\pause
\centering{\textbf{TL;DR: yup.}}
\end{frame}

\subsubsection{Fooling Set Method}

\begin{frame}{Fooling Set Method (Jake)}
Consider a two-party protocol for determining whether two inputs are equal:
\pause
\begin{example}[$Equality$]
\(EQ : \mathbb{B}^n \times \mathbb{B}^n \to \mathbb{B}\)
is precisely
\(\langle x, y \rangle \mapsto 1\)
if $x = y$ else 0.
\end{example}
\pause
Example protocol $\Pi$:\\
\begin{leftbubbles}
\texttt{P1}$=x$.
\end{leftbubbles}
\pause
\begin{rightbubbles}
\texttt{P2}$=1 \text{ if } y = x \text{ else } 0$.
\end{rightbubbles}
\end{frame}

\begin{frame}{Fooling Set Method (Jake)}
We begin with a motivating observation.
\pause
\begin{lemma}[Communication Equality is Image Equality]
If Alice and Bob exchange the same sequence of messages when Alice gets $\textbf{x}$ and Bob gets $\textbf{y}$ as they do when Alice gets $\textbf{x}'$ and Bob gets $\textbf{y}'$, then $f(\textbf{x}, \textbf{y}) = f(\textbf{x}', \textbf{y}')$.
\end{lemma}
\pause
\begin{proof}
$\Pi$ is deterministic and $f$ is a function.
\end{proof}
\pause
\underline{Idea:} an efficient protocol will efficiently group together inputs that go to the same output.
\end{frame}

\begin{frame}{Fooling Set Method (Jake)}
\underline{Idea:} an efficient protocol will efficiently \textbf{group together inputs that go to the same output.}

\begin{definition}[Fooling Set]
If $f : \mathbb{B}^n \times \mathbb{B}^n \to \mathbb{B}$ is a function,
	a \emph{fooling set} for $f$ is a set
	\(S \subseteq \mathbb{B}^n \times \mathbb{B}^n\)
	such that for some choice $b \in \mathbb{B}$
\(f(S) = \{ b \}\)
but, for all distinct $(x, y), (x', y') \in S$,
\((\neg b) \in f(\{ x, x' \} \times \{ y, y' \})\).
\end{definition}
Basically, a fooling set is a group of inputs that go to the same output,
but which is \emph{brittle} to argument-swapping.
In some sense these \emph{brittle} sets lower-bound the difficulty in grouping like inputs.
\pause
\begin{lemma}[Fooling Set Method]
If $f$ has a size-$M$ fooling set, then $C(f) \geq \text{log}_2(M)$.
\end{lemma}
\end{frame}

\begin{frame}{Fooling Set Method (Jake)}
\textbf{NTS:} If $f$ has a size-$M$ fooling set then $C(f) \geq \text{log}_2(M)$.
\begin{proof}
For a contradiction suppose a protocol $\Pi$ exists for $f$ s.t. $C(\Pi) < \text{log}_2(M)$.
\pause
Then $\Pi$ yields at most $2^{C(\Pi)} < 2^{\text{log}_2(M)} = M$
distinct communication patterns.
\pause
However, there are $M$ input pairs $(x, y)$ that each have distinct communication patterns by definition.
\pause
Since $2^{C(\Pi)} < M$ there must be some $(x, y), (x', y')$ on which $\Pi$ yields the same communication pattern.

\medskip

\pause
Then $(x, y')$ must yield the same communication pattern as $(x, y)$
as Bob cannot possibly tell the difference.  The argument is symmetric for $(x', y)$ and Alice.  One of the two must yield a contradiction and we are done.
\end{proof}
\end{frame}

\begin{frame}{Fooling Set Method (Jake)}
\begin{example}[Set-Disjointness]
$\textsc{Disj} : \mathbb{B}^n \times \mathbb{B}^n \to \mathbb{B}$ is the function
that maps $(A, B)$ to 1 if $A \cap B = \emptyset$ else 0.
\end{example}
How many fooling sets does \textsc{Disj} have?
\pause
Notice $A, B$ are disjoint iff $A \oplus B$ is \textbf{1}.
\pause
There are $2^n$ possible values $A$.
\pause
Hence $2^n$ values $(A, B)$ s.t. $A \oplus B = \textbf{1}$.
None of these distinct $(A, B), (A', B')$ satisfy
 $A \oplus B = A \oplus B'$ or
$A \oplus B = A' \oplus B$
else they wouldn't be distinct.
\pause
So we get a $2^n$-size fooling set.
\pause
\[\therefore \, C(\textsc{Disj}) \geq \text{log}_2(2^n) = n\]
\end{frame}

\subsubsection{Tiling Method}

\begin{frame}{Tiling Method (Max)}
With the \emph{fooling set} method, we lower-bounded $C(f)$.
Now we'll introduce a new method that both lower- and upper-bounds $C(f)$.
\pause
\begin{definition}[$M(f)$]
The \emph{matrix of }$f$,
	denoted $M(f)$,
		is the $2^n \times 2^n$ matrix whose $(x, y)$th entry
		is the value $f(x, y)$.
\end{definition}
\pause
\begin{example}[$M(\lor)$]
\adjustbox{max width=.25\textwidth}{
	\begin{tabular}{llllll}
	   & {\cellcolor{blue!20}{00}} & 
	     {\cellcolor{blue!20}{01}} & 
	     {\cellcolor{blue!20}{10}} & 
	     {\cellcolor{blue!20}{11}} \\
	{\cellcolor{green!20}{00}} & 00 & 01 & 10 & 11 \\
	{\cellcolor{green!20}{01}} & 01 & 01 & 11 & 11 \\
	{\cellcolor{green!20}{10}} & 10 & 11 & 10 & 11 \\
	{\cellcolor{green!20}{11}} & 11 & 11 & 11 & 11
	\end{tabular}
}
\adjustbox{max width=.73\textwidth}{
\begin{minipage}{\linewidth}
\begin{itemize}
	\item The green cells are Alice's possible inputs $x$.
	\item The blue cells are Bob's possible inputs $y$.
	\item The uncolored cells are the matrix $M(f)$.
\end{itemize}
\end{minipage}
}
\end{example}
\end{frame}

\begin{frame}{Tiling Method (Max)}
\begin{definition}[Combinatorial Rectangle]
A \emph{combinatorial rectangle} in $M(f)$ is any submatrix of $M$.
We say a rectangle $A \times B$ in $M(f)$ is \emph{monochromatic}
if for all $x, x'$ in $A$ and $y, y'$ in $B$, $M_{x,y} = M_{x',y'}$.
\end{definition}
\pause
\medskip

\underline{Idea:} Each event in a protocol $\Pi$ splits $M(f)$ into two or more
combinatorial rectangles of still-possible values for $f(x, y)$.
\pause
\medskip

\underline{Intuition:} Much like splitting a circuit $C$ into ``$C$ where the first bit is 0'' and ``$C$ where the first bit is 1''.
\pause
\medskip

Let's see an example ...
\end{frame}

\begin{frame}{Tiling Method (Max)}
\begin{example}[$\Pi=\textsc{LeastSignificantBit},f=<$]
\begin{itemize}
\item $f : \mathbb{B}^3 \times \mathbb{B}^3 \to \mathbb{B}$ is the function that maps $(x, y)$ to 1 if $x < y$ else 0.  
\item $\Pi = \textsc{LeastSignificantBit}$ is the na{\"i}ve protocol where Alice and Bob read off their bits from right to left.
\end{itemize}
\adjustbox{max width=.8\textwidth}{
\centering
\begin{forest}
[{\begin{tabular}{|l|lllll|}
\hline
    & \texttt{000}
    & \texttt{001}
    & \texttt{010}
    & \texttt{011} 
    & \texttt{100}\\\hline
\texttt{000} & \cellcolor{red!25}0 & 
                     \cellcolor{green!25}1 & 
                     \cellcolor{green!25}1 & 
                     \cellcolor{green!25}1 & 
                     \cellcolor{green!25}1 \\
\texttt{001} & \cellcolor{red!25}0 & 
                     \cellcolor{red!25}0 & 
                     \cellcolor{green!25}1 & 
                     \cellcolor{green!25}1 & 
                     \cellcolor{green!25}1 \\
\texttt{010} & \cellcolor{red!25}0 & 
                     \cellcolor{red!25}0 & 
                     \cellcolor{red!25}0 & 
                     \cellcolor{green!25}1 & 
                     \cellcolor{green!25}1 \\
\texttt{011} & \cellcolor{red!25}0 & 
                     \cellcolor{red!25}0 & 
                     \cellcolor{red!25}0 & 
                     \cellcolor{red!25}0 & 
                     \cellcolor{green!25}1 \\
\texttt{100} & \cellcolor{red!25}0 & 
                     \cellcolor{red!25}0 & 
                     \cellcolor{red!25}0 & 
                     \cellcolor{red!25}0 & 
                     \cellcolor{red!25}0\\\hline
\end{tabular}},
	[{\begin{tabular}{|l|lllll|}
\hline
    & \texttt{000}
    & \texttt{001}
    & \texttt{010} 
    & \texttt{011}
    & \texttt{100}\\\hline
\texttt{000}  & \cellcolor{red!25}0 & 
                     \cellcolor{green!25}1 & 
                     \cellcolor{green!25}1 & 
                     \cellcolor{green!25}1 & 
                     \cellcolor{green!25}1 \\
\texttt{010}  & \cellcolor{red!25}0 & 
                     \cellcolor{red!25}0 & 
                     \cellcolor{red!25}0 & 
                     \cellcolor{green!25}1 & 
                     \cellcolor{green!25}1 \\
\texttt{100}  & \cellcolor{red!25}0 & 
                     \cellcolor{red!25}0 & 
                     \cellcolor{red!25}0 & 
                     \cellcolor{red!25}0 & 
                     \cellcolor{red!25}0\\\hline
\end{tabular}}, edge label={node[midway,left]{Alice: ``$x=\_\_\texttt{0}$''}}]
	[{\begin{tabular}{|l|lllll|}
\hline
    & \texttt{000} 
    & \texttt{001}
    & \texttt{010}
    & \texttt{011}
    & \texttt{100}\\\hline
\texttt{001}  & \cellcolor{red!25}0 & 
                     \cellcolor{red!25}0 & 
                     \cellcolor{green!25}1 & 
                     \cellcolor{green!25}1 & 
                     \cellcolor{green!25}1 \\
\texttt{011}  & \cellcolor{red!25}0 & 
                     \cellcolor{red!25}0 & 
                     \cellcolor{red!25}0 & 
                     \cellcolor{red!25}0 & 
                     \cellcolor{green!25}1 \\\hline
\end{tabular}}, edge label={node[midway,right]{Alice: ``$x=\_\_\texttt{1}$''}}]]
\end{forest}}
\end{example}
\end{frame}

\begin{frame}{Tiling Method (Max)}
Now we get to the punchline.
\pause
\begin{definition}[Monochromatic Tiling]
A \emph{monochromatic tiling} of $M(f)$ is a partition of $M(f)$ into disjoint
monochromatic rectangles.
\end{definition}
\pause
\begin{center}
\textbf{It's thinking time.}
\end{center}
\pause
\begin{itemize}
\item Then the leaves of the tree induced by $\Pi$ and rooted at $M(f)$ clearly form a monochromatic tiling of $M(f)$.
\pause
\item The number of leaves in a binary tree can be used to upper-bound its depth.
\pause
\item The depth of the binary tree induced by $\Pi$ is exactly $C(\Pi)$.
\end{itemize}
\end{frame}

\begin{frame}{Tiling Method (Max)}
\begin{center}
\textbf{It's thinking time.}
\end{center}
\begin{itemize}
\item Then the leaves of the tree induced by $\Pi$ and rooted at $M(f)$ clearly form a monochromatic tiling of $M(f)$.
\item The number of leaves in a binary tree can be used to upper-bound its depth.
\item The depth of the binary tree induced by $\Pi$ is exactly $C(\Pi)$.
\end{itemize}
\pause
\begin{theorem}[The Punchline]
Let $\chi(f)$ denote the minimum number of rectangles in any monochromatic tiling of $M(f)$.
\[\text{log}_2 \chi(f) \leq
	C(f) \leq
		16 \big( \text{log}_2 \chi(f) \big)^2
\]
\end{theorem}
\end{frame}

\begin{frame}{Tiling Method (Max)}
\textbf{NTS:} $\text{log}_2 \chi(f) \leq C(f)$.
\begin{proof}
Assume $C(f)$. 
\pause
Then $\exists$ a protocol $\Pi$ in which $\leq C(f)$ bits are communicated
between the 2 participants. 
\pause 
For simplicity suppose each bit is communicated individually. 
\pause 
Then $\Pi$ induces a tree whose max depth is $C(f)$, whose leaves form a monochromatic partition of $M(f)$.
\pause  
Every m.c. partition / $M(f)$ requires $\geq \chi(f)$ rectangles, so the tree induced by $\Pi$ has $\geq \chi(f)$ leaves.
\pause  
But it's a binary tree so its depth is at least $\text{log}_2 \chi(f)$.
\end{proof}
\end{frame}

% This proof is *enormous*, and appears to contain
% a small arithmetic error somewhere near the end 
% (a missing 2 coefficient).  So, we probably should not
% present it, but there's no harm in including it in the
% supplemental materials and sharing it online.
\begin{frame}{Tiling Method (Max)}
{\small \textbf{NTS:} $C(f) \leq 16 \big( \text{log}_2 \chi(f) \big)^2$.~\cite{aho1983notions}}
\begin{proof}
\only<1>{
\begin{itemize}
	\item Let $M_1, ..., M_{\chi(f)}$ be a monochromatic
	partitioning of $M(f)$ known ahead of time to both
	Alice (on the ``left'') 
	and Bob (on the ``right'').
	Each rectangle $M_i$ can alternatively be written $X_i \times Y_i$.
	\item Let $G_L, G_R$ be graphs whose nodes are $\{ 1, ..., \chi(f) \}$.
	There is an edge $i \to j$ in $G_L$ ($G_R$ resp.)
		if $M_i$ and $M_j$ have a row (column resp.) in common.
	\item Let $\text{deg}_L(u)$ (resp. $\text{deg}_R(u)$)
		denote the degree of the node $u$ in the graph $G_L$
		(resp. $G_R$.)
	\item Let $x$ be Alice's input and $y$ Bob's input.
\end{itemize}
}
\only<2>{
\begin{itemize}
	\item We'll describe the protocol in ``rounds''.
	During the rounds, Alice keeps track of $Y$ (a set containing $y$)
	and Bob keeps track of $X$ (a set containing $x$),
	both of which are initially $\mathbb{B}^n$.
	\item Both sides know the graphs $G_L, G_R$ and the rectangles $M_i$
	ahead of time.
\end{itemize}
}
\only<3>{
Each stage proceeds as follows.
\begin{enumerate}
	\item Alice looks for a rectangle $M_i = X_i \times Y_i$ s.t. $x \in X_i$ and $\text{deg}_L(i) \leq 3\chi(f)/4$.
	\begin{enumerate}
		\item If she finds some such rectangle then she sends $i$ to Bob.
		\begin{enumerate}
			\item Bob replies to indicate if $y \in M_i$.  
			\item If so then the protocol ends because $f(x, y)$ is the color of $M_i$.
			\item Otherwise $X := X \cap X_i$, and each rectangle $M_{\alpha} = X_{\alpha} \cap Y_{\alpha}$ is replaced with $(X_i \cap X_{\alpha}) \times Y_{\alpha}$.
		\end{enumerate}
		\item Otherwise she replies that she found no such rectangle.
		In this case Bob does what Alice just attempted, symmetrically, with a small caveat ...
	\end{enumerate}
\end{enumerate}
}
\only<4>{
\begin{itemize}
\item If neither Alice nor Bob could find any $M_i$ with low-enough degree,
	then they both know that every node $i$ in $G := G_L \cap G_R$
	for which $(x, y) \in M_i$
	has degree $\geq (3\chi(f)/4)^2$ $= 9\chi(f)/16$ $> \chi(f)/2$.
\item Let $i, j$ both have degree $\geq \chi(f)/2$ in $G$.
	Then some node $z$ is adjascent to $i$ and $j$ in $G$, 
	by the Pigeonhole Principle.
	Hence $M_i \cap M_z = \emptyset$ and $M_j \cap M_z = \emptyset$.
	But the rectangles are monochromatic, hence, $M_i$ and $M_j$ are the same color.  So Alice needs to find an $M_i$ containing $x$ and some $y \in Y$ whose degree in $G$ is at least $\chi(f)/2$; and Bob's procedure is symmetric.
\end{itemize}
}
\only<5>{
\begin{itemize}
	\item In the worst case for each stage, the first participant sends ``nothing found'' (1 bit), the second participant sends some $i$ ($\leq \text{log}_2(\chi(f))$ bits), and the first participant replies with some $j$ ($\leq \text{log}_2(\chi(f))$ bits).  So in the worst case each round requires $\leq 2 + 2\text{log}_2(\chi(f))$ bits.
	\item The protocol ends after at most $n$ rounds where $(3\chi(f)/4)^n \approx 1$, i.e., after $\text{log}_{(4/3)}(\chi(f))$ rounds.
	\item So total communication complexity is $\leq \text{log}_{(4/3)}(\chi(f)) * (1 + 2\text{log}_2(\chi(f)))$.
\end{itemize}
}
\only<6>{
	For $\chi(f) \geq 2$:
	\[\begin{aligned}
	C(f) & \leq \text{log}_{(4/3)}(\chi(f)) * (1 + 2\text{log}_2(\chi(f))) \\
		 & = \frac{\text{log}_2(\chi(f))}{\text{log}_2(4/3)} * (1 + 2\text{log}_2(\chi(f))) \\
		 & < 2.5 * \text{log}_2(\chi(f)) * (1 + 2\text{log}_2(\chi(f))) \\
		 & \leq 2.5 * \text{log}_2(\chi(f)) * 3\text{log}_2(\chi(f))) \\
		 & = 7.5 \text{log}_2^2(\chi(f)) \\
		 & \leq 16 \text{log}_2^2(\chi(f))
	\end{aligned}\]
	{\tiny I'm almost certainly missing a factor of 2, which would explain the choice of 16. -Max.}
}
\alt<6>{\qedhere}{\phantom\qedhere}
\end{proof}

\end{frame}

\subsubsection{Discrepency Method}

\begin{frame}{2-Party Discrepency Method (Max)}
Recall that $\chi(f)$ induces both lower and upper bounds on $C(f)$.
So if any bound on $\chi(f)$ induces a bound on $C(f)$.
We are about to prove the following lower-bound on $\chi(f)$:
\[\begin{aligned}\text{Disc}(A \times B) 
	& = \frac{1}{2^n * 2^n}\abs{\sum_{a \in A, b \in B} (-1)^{M_{a,b}}} \\
	& \leq \chi(f)
\end{aligned}\]
\end{frame}

\begin{frame}{2-Party Discrepency Method (Max)}
When we partition $M(f)$ into some number of rectangles,
	the sizes of the rectangles must add up to the size of $M(f)$.
\pause
\medskip

Hence, if $\chi(f) \leq K$ for some integer $K$,
	then $M(f)$ must have a m.c. rectangle containing at least $2^n * 2^n / K$ entries.
\pause
\begin{proof}
Suppose $\chi(f) \leq K$ for some integer $K$.
\pause
If $\chi(f) = K$ then $\exists$ a partioning of $M(f)$ into $K$ m.c. rects, in which case at least 1 must have size $\geq \abs{M(f)} / K$, i.e., $2^n * 2^n / K$.  
\pause
On the other hand if $\chi(f) < K$ then $\chi(f) = K'$ for some $K' < K$ and then $M(f)$ can be partitioned into $K'$ monochromatic rectangles, at least 1 of which has size $\geq \abs{M(f)} / K'$, which is strictly larger than $\abs{M(f)} / K$.
\pause
Either way the conjecture holds.
\end{proof}
\end{frame}

\begin{frame}{2-Party Discrepency Method (Max)}
Suppose that $M(f)$ contains a monochromatic rectangle $A \times B$ having at least $2^n * 2^n / K$ entries.
\pause
Since $A \times B$ is monochromatic, this implies that:
\[\sum_{a \in A, b \in B} (-1)^{M_{a,b}} = \begin{cases}
-1 * \abs{A \times B} & \text{ if it's colored } 1 \\
+1 * \abs{A \times B} & \text{ if it's colored } 0\end{cases}\]
\pause
So if we wrap an absolute value above our sum, we get:
\[\abs{\sum_{a \in A, b \in B} (-1)^{M_{a,b}}} = \text{ the size of the rectangle } A \times B\]
\pause
But we already assumed that $A \times B$ has at least $2^n * 2^n / K$ entries, hence:
\[\abs{\sum_{a \in A, b \in B} (-1)^{M_{a,b}}} = \geq 2^n * 2^n / K\]
\end{frame}

\begin{frame}{2-Party Discrepency Method (Max)}
Let's divide both size by $2^n * 2^n$, for fun and profit.
\[\frac{1}{2^n * 2^n}\abs{\sum_{a \in A, b \in B} (-1)^{M_{a,b}}} \geq 1 / K\]
\pause
Mathematicians like to name things.
\begin{definition}[Discrepency]
The \textit{discrepency} of a rectangle $A \times B$ of $M(f)$ is exactly the following.
\[\text{Disc}(A \times B) = \frac{1}{2^n * 2^n}\abs{\sum_{a \in A, b \in B} (-1)^{M_{a,b}}}\]
The \emph{discrepency of $M(f)$} is the max disc among its rectangles.
\end{definition}
\end{frame}

\begin{frame}{2-Party Discrepency Method (Max)}
Now that we've named this thing, let's re-write our inequality.
\[\text{Disc}(A \times B) \geq 1 / K\]
\pause
Taking inverses:
\[\frac{1}{\text{Disc}(A \times B)} \leq K\]
\pause
Certainly $\chi(f) \leq \chi(f)$, so supplanting $\chi(f)$ for $K$ in the statement, we get:
\begin{lemma}[2-Party Discrepency Method]
\[\frac{1}{\text{Disc}(A \times B)} \leq \chi(f)\]
\end{lemma}
\end{frame}

\subsection{Multi-Party Problem}

\begin{frame}{Multi-Party Problem (Jake)}
\begin{itemize}
\item There are $k$ of us.
\item We all place a sticky note with some value $b \in \mathbb{B}^n$ on our heads. 
\item Without talking, we must compute some predetermined function via a predetermined protocol. All communication must be done through the whiteboard in front of us.
\item The goal is for one player, after some amount of communication, to write the value $f(\text{sticky-note}_1, ..., \text{sticky-note}_k)$ on the whiteboard.
\end{itemize}
\end{frame}

\begin{frame}{Multi-Party Problem (Jake)}
\begin{example}[Majority Parity]
\(\textsc{MajPar} : \mathbb{B}^n \times \mathbb{B}^n \times \mathbb{B}^n \to \mathbb{B}\)
is precisely
\(\langle x_{1}, x_{2}, x_{3} \rangle \mapsto 1\)
if $\bigoplus_{i = 1}^{n} \text{majority}(x_{1i}, x_{2i}, x_{3i})$ else 0.
\end{example}
\pause
For example: $f(1101, 1001, 1011) = \bigoplus 1001 = 0$\\
\pause
Example protocol $\Pi$:\\

\begin{tabular}{|lll|}
  \hline
  Player 1 & Player 2 & Player 3 \\ [0.5ex] 
  \hline
  $x_{2}=1001$                                    & $x_{1}=1101$          & $x_{1}=1101$         \\
  $x_{3}=1011$                                    & $x_{3}=1011$          & $x_{2}=1001$         \\
                                                  &                       &                      \\
  $\text{parity}(1 0 \_ 1)$                       & $\text{parity}(1 \_ \_ 1)$   
                                                  & $\text{parity}(1 \_ 0 1)$ \\
  $p_{1} = 0$                                     & $p_{2} = 0$          
                                                  & $p_{3} = 0$          \\\hline
  $\text{parity}(p_{1}p_{2}p_{3}) = \text{parity}(000) = 0$  &  & \\\hline
\end{tabular}

\end{frame}

\begin{frame}{Multi-Party Problem (Jake)}
Before we talked about \emph{rectangles}.
Now: \emph{cylinders}.
\begin{definition}[Cylinder]
A \textit{cylinder in dimension $i$} is a subset $S$ of the inputs $(\mathbb{B}^n)^k$ such that if $(x_1, ..., x_{i - 1}, x_i, x_{i + 1}, ..., x_k) \in S$,
then for all $x_i' \in \mathbb{B}^n$, so is $(x_1, ..., x_{i - 1}, x_i', x_{i + 1}, ..., x_k)$.
\end{definition}
\pause
\begin{definition}[Cylinder Intersection]
A \emph{cylinder intersection} is a set
\(C = \bigcap_{i = 1}^k T_i\)
where each $T_i$ is a cylinder in dimension $i$.
\end{definition}
\pause
\begin{lemma}[Generalize $\chi(f)$]
If every partition of $M(f)$ into m.c. cylinder intersections requires at least $R$ of them, then $C(f) \geq \ceil{\text{log}_2(R)}$.
\end{lemma}
\end{frame}

\subsubsection{Multi-Party Discrepency Method}

\begin{frame}{Multi-Party Discrepency Method (Max)}
\begin{definition}[Multi-Party Discrepency]
Suppose
\[f : \underbrace{\mathbb{B}^n \times ... \times \mathbb{B}^n}_{k \text{ times}} \to \mathbb{B}\]
is a function.  Then the \emph{$k$-party discrepency of $f$} is defined as follows, where $T$ ranges over all cylinder intersections of $f$.
\[\text{Disc}(f) = \frac{1}{(2^n)^k} \max_T \abs{\sum_{(x_1,...,x_k) \in T} f(x_1,...,x_k)}\]
\end{definition}
\pause
Man this is really complicated.
\pause
Could we lower-bound it statistically?
\end{frame}

\begin{frame}{Multi-Party Discrepency Method (Max)}
First some extremely tedious definitions.
\pause
\begin{definition}[$(k, n)$-Cube]
A $(k, n)$-cube is a set $D$ of the form
\(D = \{ a_1, a_1' \} \times ... \times \{ a_k, a_k' \}\)
where each $a_i, a_i' \in \mathbb{B}^n$.
A point $\vec{d} \in D$ is a vector $(x_1, x_2, ..., x_k)$ s.t. each $x_i \in \{ a_i, a_i' \}$.
\end{definition}
\pause
\begin{definition}[$\mathcal{E}$]
Let $f : (\mathbb{B}^n)^k \to \mathbb{B}$ be a function.
\[\mathcal{E}(f) = \underset{\substack{D \text{ is a}\\(k, n)\text{-cube}}}{{\LARGE \text{E}}} \Big[\prod_{\vec{d} \in D} f(\vec{d}) \Big]\]
I.e., $\mathcal{E}(f) = E[$\emph{given an arbitrary cube $D$, what is the product of the image of $f$ over all the points $\vec{d} \in D$?}$]$
\end{definition}
\end{frame}

\begin{frame}{Multi-Party Discrepency Method (Max)}
Although somewhat scary-looking, this definition pays dividends immediately.
\begin{lemma}[$k$-Party Discrepency Bound]
If $f : (\mathbb{B}^n)^k \to \mathbb{B}$ is a function,
then $\text{Disc}(f) \leq (\mathcal{E}(f))^{1/2^k}$.
\end{lemma}
\pause
Proof sketch:
\begin{proof}
{\tiny Given any cylinder intersection and $(n, k)$-cube, what is the expectation / the image of the cube?  What if we only consider points in the cylinder intersection? Derive a lower bound on $\mathcal{E}(f)$ like}
\[{\scriptstyle \mathcal{E}(f) \geq E_{x_1,...,x_k}[f(x_1,....,x_k)(1 \text{ if } (x_1,...,x_k) \in C \text{ else } 0)]^{2k}}\]
{\tiny given a cylinder intersection $C$. Argue from the def. of the $k$-party discrepency that this gives a natural lower-bound $\mathcal{E}(f) \geq \text{Disc}(f)^{2k}$. But this implies \(\text{Disc}(f) \leq (\mathcal{E}(f))^{1/2^k}\), and we're done.}
\end{proof}
\end{frame}

\section{Other Variants}

\subsection{Non-Deterministic}

\begin{frame}{Non-Deterministic (Jake)}
\begin{itemize}
	\item Defined similarly to $\NP$.
	\pause
	\item Consider a two party problem (we can generalize to multi-party from here).  Each player is given their input along with some nondeterministic guess $z$ of length $m$ that may depend on the given inputs.  $C(f) = m + \text{communication}$, and $f(x, y) = 1$ iff $\exists z$ that makes the players output 1.
	\pause
	\item $\NP^{CC}$ is the class of nondeterministic functions $f$ s.t. $C(f) = n^{k}$.  $\coNP^{CC}$ is defined similarly, i.e., $g(x, y) = 1 - f(x, y)$ for $f \in \NP^{CC}$.
	\pause
	\item Claim: $\NP^{CC} \cap \coNP^{CC} = \Pcl^{CC}$.
	This is shown by relating the communication complexities of $f \in \NP^{CC}$ and $\overline{f} \in \coNP^{CC}$.
	\pause
	\item $C(f) = k$, and $C(\overline{f}) = 10kl$ for some complexity $l$.
\end{itemize}
\end{frame}

\subsection{Randomized}

\begin{frame}{Randomized (Max)}
\begin{center}\textbf{New Rules!}\end{center}

\begin{figure}
  \includegraphics[width=0.3\linewidth]{coin.png}
  \caption{A coin.}
\end{figure}

\begin{itemize}
	\item You're allowed to be wrong sometimes.
	\item You have a lava lamp or a coin or something.
	\item Instead of $C(f)$ the complexity of $f$,
	we discuss $R(f) = \mathbb{E}[C(f)]$ the \emph{expected}
	complexity of $f$.
\end{itemize}
\end{frame}

\begin{frame}{Randomized (Max)}
\begin{center}\textbf{Two Models}\end{center}
\pause
\begin{itemize}
	\item \textbf{Public Coin:} Everyone can see the same random string ahead of time.
	\pause
	\item \textbf{Private Coin:} Everyone can flip their own coin(s) in private.
\end{itemize}

\end{frame}

\section{References}

\begin{frame}[t, allowframebreaks]{References}
\bibliographystyle{amsalpha}
\bibliography{pres}
\end{frame}

\end{document}