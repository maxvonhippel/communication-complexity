\title{Fooling Sets}
\author{Jake Kinsella}
\date{\today}

\documentclass[12pt]{article}

\usepackage{usenixy}
\usepackage{
  amsmath,mathtools,amssymb,fullpage,
  amsthm,centernot,float,diagbox,adjustbox,
  graphicx,caption,subcaption,listings,
  enumerate,xcolor,hyperref,color,soul,
  stmaryrd,amsrefs,titling,tikz,
  mathrsfs,etoolbox,refcount,multicol,MnSymbol,
  yfonts,framed,relsize,colortbl,pgf-umlsd,varwidth,
  environ,xparse}
\lstset{
  language=Lisp,
  basicstyle=\ttfamily,
}
\usepackage[many]{tcolorbox}


\usepackage[framemethod=tikz]{mdframed}

\newcommand{\grayleftbar}[1]{\textcolor[gray]{0.7}{%
     \begin{leftbar}\textcolor{black}{#1}\end{leftbar}}}   
\newcommand{\darkleftbar}[1]{\textcolor[gray]{0.3}{%
     \begin{leftbar}\textcolor{black}{#1}\end{leftbar}}}   

\newcounter{countitems}
\newcounter{nextitemizecount}
\newcommand{\setupcountitems}{%
  \stepcounter{nextitemizecount}%
  \setcounter{countitems}{0}%
  \preto\item{\stepcounter{countitems}}%
}
\makeatletter
\newcommand{\computecountitems}{%
  \edef\@currentlabel{\number\c@countitems}%
  \label{countitems@\number\numexpr\value{nextitemizecount}-1\relax}%
}
\newcommand{\nextitemizecount}{%
  \getrefnumber{countitems@\number\c@nextitemizecount}%
}
\newcommand{\previtemizecount}{%
  \getrefnumber{countitems@\number\numexpr\value{nextitemizecount}-1\relax}%
}
\makeatother    
\newenvironment{AutoMultiColItemize}{%
\ifnumcomp{\nextitemizecount}{>}{3}{\begin{multicols}{2}}{}%
\setupcountitems\begin{itemize}}%
{\end{itemize}%
\unskip\computecountitems\ifnumcomp{\previtemizecount}{>}{3}{\end{multicols}}{}}

\usepackage{titlesec}
\usetikzlibrary{
    arrows,shapes,positioning,decorations.markings,
    decorations.pathreplacing,hobby,automata,tikzmark}
\tikzset{
    looped/.style={
        decoration={markings,mark=at position 0.999 with {\arrow[scale=2]{>}}},
        postaction={decorate},
        >=stealth
    },
    straight/.style={
        decoration={markings,mark=at position 1 with {\arrow[scale=2]{>}}},
        postaction={decorate},
        >=stealth
    },
    loopedSF/.style={
        decoration={
            markings,
            mark=at position 0.999 with {\arrow[scale=2]{>}},
            mark=at position 0.5 with {\arrow[scale=2]{>}}},
        postaction={decorate},
        >=stealth
    },
    straightSF/.style={
        decoration={
            markings,
            mark=at position 0.999 with {\arrow[scale=2]{>}},
            mark=at position 0.5 with {\arrow[scale=2]{>}}},
        postaction={decorate},
        >=stealth
    },
    triangle/.style = {fill=white, draw=black, regular polygon, regular polygon sides=3 },
    initial text=$$, % sets the text that appears on the start arrow
    node distance=1.5cm, % specifies the minimum distance between two nodes. Change if necessary.
    ->, % makes the edges directed
}
\titleformat{\subsection}[runin]
{\normalfont\large\bfseries}{\thesubsection}{1em}{}

\titleformat{\subsubsection}[runin]
{\normalfont\normalsize\bfseries}{\thesubsubsection}{1em}{}



\definecolor{mycolor}{rgb}{0.122, 0.435, 0.698}

\newmdenv[innerlinewidth=0.5pt, roundcorner=4pt,linecolor=mycolor,innerleftmargin=6pt,
innerrightmargin=6pt,innertopmargin=6pt,innerbottommargin=6pt]{mybox}

\usepackage[pdf]{graphviz}
\usetikzlibrary{shapes,arrows,calc}
\usepackage[utf8]{inputenc}

\setlength{\droptitle}{-7em}
\posttitle{\par\end{center}}

\newtheorem{definition}{Definition}
\newtheorem{theorem}{Theorem}
\newtheorem{lemma}{Lemma}
\newtheorem{problem}{Problem}
\newtheorem{example}{Example}
\newtheorem{remark}{Remark}

\newcommand{\TODO}[0]{\textcolor{red}{TODO}}
\newcommand{\F}[0]{\mathsf{F}}
\newcommand{\U}[0]{\mathsf{U}}
\newcommand{\G}[0]{\mathsf{G}}
\newcommand{\X}[0]{\mathsf{X}}
\newcommand{\acs}[0]{\textsc{ACL2S}}
\newcommand{\acl}[0]{\textsc{ACL2}}
\newcommand{\pr}[0]{\textsc{Promela}}
\newcommand{\ko}[0]{\textsc{Korg}}
\newcommand{\spn}[0]{\textsc{SPIN}}
\newcommand{\ie}{\emph{i.e.}}
\newcommand{\etal}{\emph{et al.}}

\newcommand{\Pcl}[0]{\textsc{P}}
\newcommand{\NP}[0]{\textsc{NP}}
\newcommand{\SAT}[0]{\texttt{SAT}}
\newcommand{\coNP}[0]{\textsc{coNP}}
\newcommand{\Time}[0]{\text{TIME}}
\newcommand{\Space}[0]{\text{SPACE}}
\newcommand{\threeSAT}[0]{\texttt{3SAT}}

\DeclarePairedDelimiter\abs{\lvert}{\rvert}
\DeclarePairedDelimiter\norm{\lVert}{\rVert}

\usepackage{tikz}
\usetikzlibrary{arrows,chains,matrix,positioning,scopes}

\newlength{\bubblewidth}
\AtBeginDocument{\setlength{\bubblewidth}{.75\textwidth}}
\definecolor{bubblegreen}{RGB}{103,184,104}
\definecolor{bubblegray}{RGB}{241,240,240}

\newcommand{\bubble}[4]{%
  \tcbox[
    colback=#1,
    colframe=#1,
    #2,
  ]{\color{#3}\begin{varwidth}{\bubblewidth}#4\end{varwidth}}%
}

\ExplSyntaxOn
\seq_new:N \l__ooker_bubbles_seq
\tl_new:N \l__ooker_bubbles_first_tl
\tl_new:N \l__ooker_bubbles_last_tl

\NewEnviron{rightbubbles}
 {
  \raggedleft\sffamily
  \seq_set_split:NnV \l__ooker_bubbles_seq { \par } \BODY
  \int_compare:nTF { \seq_count:N \l__ooker_bubbles_seq < 2 }
   {
    \bubble{bubblegreen}{rounded~corners}{white}{\BODY}
   }
   {
    \seq_pop_left:NN \l__ooker_bubbles_seq \l__ooker_bubbles_first_tl
    \seq_pop_right:NN \l__ooker_bubbles_seq \l__ooker_bubbles_last_tl
    \bubble{bubblegreen}{sharp~corners=southeast}{white}{\l__ooker_bubbles_first_tl}\par
    \seq_map_inline:Nn \l__ooker_bubbles_seq
     {
      \bubble{bubblegreen}{sharp~corners=east}{white}{##1}\par
     }
    \bubble{bubblegreen}{sharp~corners=northeast}{white}{\l__ooker_bubbles_last_tl}\par
   }
 }
\NewEnviron{leftbubbles}
 {
  \raggedright\sffamily
  \seq_set_split:NnV \l__ooker_bubbles_seq { \par } \BODY
  \int_compare:nTF { \seq_count:N \l__ooker_bubbles_seq < 2 }
   {
    \bubble{bubblegray}{rounded~corners}{black}{\BODY}
   }
   {
    \seq_pop_left:NN \l__ooker_bubbles_seq \l__ooker_bubbles_first_tl
    \seq_pop_right:NN \l__ooker_bubbles_seq \l__ooker_bubbles_last_tl
    \bubble{bubblegray}{sharp~corners=southwest}{black}{\l__ooker_bubbles_first_tl}\par
    \seq_map_inline:Nn \l__ooker_bubbles_seq
     {
      \bubble{bubblegray}{sharp~corners=west}{black}{##1}\par
     }
    \bubble{bubblegray}{sharp~corners=northwest}{black}{\l__ooker_bubbles_last_tl}\par
   }
 }
\ExplSyntaxOff

\begin{document}
\maketitle

Consider a two-party protocol for determining whether two inputs are equal:
\begin{equation}
  EQ(x, y) =
  \begin{cases}
    1 \text{ if x=y} \\
    0 \text{ otherwise}
  \end{cases}
\end{equation}


The "obvious protocol" is for Player 1 to send it's entire input to Player 2 and let Player 2 compare the values itself.
\\

\textbf{Protocol $\Pi$}
\begin{center}
  \begin{tabular}{ |m{15em}|m{15em}| } 
    \hline
    Player 1                   & Player 2 \\ [0.5ex] 
    \hline
    $x=111$                    & $y=110$ \\
    $P_{1}(x) \longrightarrow$ &  \\
                               & $p_{1} = 111$ \\
                               & $\longleftarrow P_{2}(y, p_{1}) = P_{2}(110, 111) = 0$ \\
    $p_{2} = 0$                & \\
    \hline
  \end{tabular}
\end{center}

In this example, 4 bits are communicated. More generally the $C(\Pi) = |x| + 1$, the cost for Player 1 to communicate x
plus one bit for Player 2 to communicate the answer.
\\

\textbf{Theorem:} $C(EQ) \geq n$
\\

\textbf{Claim:} For any $(x, x)$ and $(x', x')$, if on both inputs, both Players communicate the exact same sequence of
bits, then $f(x, x) = f(x', x') = f(x, x') = f(x', x)$

\begin{multicols}{2}

  $(x, x)$
  \begin{center}
    \begin{tabular}{ |m{7em}|m{7em}| } 
      \hline
      Player 1                   & Player 2 \\ [0.5ex] 
      \hline
      $x$                        & $x$ \\
      $P_{1}(x) \longrightarrow$ &  \\
                                 & $p_{1}$ \\
                                 & $\longleftarrow P_{2}(x, p_{1})$ \\
      $p_{2}$                    & \\
      \hline
    \end{tabular}
  \end{center}

\columnbreak

  $(x', x')$
  \begin{center}
    \begin{tabular}{ |m{7em}|m{7em}| } 
      \hline
      Player 1                   & Player 2 \\ [0.5ex] 
      \hline
      $x$                        & $x$ \\
      $P_{1}(x') \longrightarrow$ &  \\
                                 & $p_{1}'$ \\
                                 & $\longleftarrow P_{2}(x', p_{1}')$ \\
      $p_{2}'$                   & \\
      \hline
    \end{tabular}
  \end{center}

\end{multicols}

$p_{1} = p_{1}'$ and $p_{2} = p_{2}'$
$p_{2} = p_{2}'$ is the final answer.
\\

If $(x, x)$ and $(x', x')$ have the same communication pattern, then it doesn't matter which way the inputs are mixed.
Each $x$ and $x'$ produce the same bits.
\\

If they produce the same sequence of bits, then they agree on the output (as the output is just the final bit).
\\

\textbf{Proof:} $C(EQ) \geq n$
\\

Assume a protocol $\Pi'$ with complexity $n - 1$ exists that solves $EQ$.
\\

$\Pi'$ has $2^{n - 1}$ (${0, 1}$ over $n - 1$ bits) possible communication patterns if it can communicate a max $n - 1$
bits.
\\

However, there are $2^{n}$ input pairs $(x, x)$ ($|x| = n$, ${0, 1}$ over $n$ bits)
\\

$\Pi'$ has $2^{n - 1}$ possible communication patterns.
However there are $2^{n}$ equal input pairs.

Thus there exists some:
$(x, x)$ and $(x', x')$ where $x \neq x'$ that have the same communication protocol
\\

This is a contradiction.
\\
$EQ(x, x') = 0 \neq EQ(x, x)$
\\

Thus: $C(EQ) \geq n$
\\

\textbf{Lemma (generalization of the above theorem)}: Given some $f : {0, 1}^{n} \times {0, 1}^{n} \rightarrow {0, 1}$. $f$
has a M-sized fooling set if there exists an M-sized subset $S \subset {0, 1}^{n} \times {0, 1}^{n}$ and a value
$b \in {0, 1}$ such that:
\par{1) $\forall (x, y) \in S, f(x, y) = b$}
\par{2) $\forall distinct (x, y), (x', y') \in S, either f(x, y') \neq b or f(x', y) \neq b$}
\\
If $f$ has a size-M fooling set, then $C(f) \geq log(M)$
\\

\textbf{Example:}
\par{Determining whether two sets are disjoint across two parties.}

\par{$x, y \subset {1, 2, ..., n} \\$}
\\
\par{
  \begin{equation}
    DISJ(x, y) =
    \begin{cases}
      1 \text{ if } x \cap y = \emptyset \\
      0 \text{ otherwise}
    \end{cases}
  \end{equation}
}

\par{Fooling set for $DISJ$:}

\par{$S = {(A, \overline{A}) | A \subset {1, 2, ..., n}}$}
\par{\hspace{\parindent} 1) $\forall A, DISJ(A, \overline{A}) = 1$}
\par{\hspace{\parindent} 2) $\forall (A, \overline{A}), (B, \overline{B}), either DISJ(A, \overline{B}) = 0 or DISJ(B, \overline{A}) = 0$}
\\
\par{There are $2^{n}$ possible A sets}
\\
\par{Thus S is a $2^{n}$-sized fooling set}
\\
\par{Therefore $C(DISJ) \geq log(M) \geq n$}

\end{document}