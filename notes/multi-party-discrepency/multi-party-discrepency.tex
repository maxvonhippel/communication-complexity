\title{The Multi-Party Discrepency Method}
\author{Max von Hippel}
\date{\today}

\documentclass[12pt]{article}

\usepackage{usenixy}
\usepackage{
  amsmath,mathtools,amssymb,fullpage,
  amsthm,centernot,float,diagbox,adjustbox,
  graphicx,caption,subcaption,listings,
  enumerate,xcolor,hyperref,color,soul,
  stmaryrd,amsrefs,titling,tikz,
  mathrsfs,etoolbox,refcount,multicol,MnSymbol,
  yfonts,framed,relsize,colortbl,pgf-umlsd,varwidth,
  environ,xparse}
\lstset{
  language=Lisp,
  basicstyle=\ttfamily,
}
\usepackage[many]{tcolorbox}


\usepackage[framemethod=tikz]{mdframed}

\newcommand{\grayleftbar}[1]{\textcolor[gray]{0.7}{%
     \begin{leftbar}\textcolor{black}{#1}\end{leftbar}}}   
\newcommand{\darkleftbar}[1]{\textcolor[gray]{0.3}{%
     \begin{leftbar}\textcolor{black}{#1}\end{leftbar}}}   

\newcounter{countitems}
\newcounter{nextitemizecount}
\newcommand{\setupcountitems}{%
  \stepcounter{nextitemizecount}%
  \setcounter{countitems}{0}%
  \preto\item{\stepcounter{countitems}}%
}
\makeatletter
\newcommand{\computecountitems}{%
  \edef\@currentlabel{\number\c@countitems}%
  \label{countitems@\number\numexpr\value{nextitemizecount}-1\relax}%
}
\newcommand{\nextitemizecount}{%
  \getrefnumber{countitems@\number\c@nextitemizecount}%
}
\newcommand{\previtemizecount}{%
  \getrefnumber{countitems@\number\numexpr\value{nextitemizecount}-1\relax}%
}
\makeatother    
\newenvironment{AutoMultiColItemize}{%
\ifnumcomp{\nextitemizecount}{>}{3}{\begin{multicols}{2}}{}%
\setupcountitems\begin{itemize}}%
{\end{itemize}%
\unskip\computecountitems\ifnumcomp{\previtemizecount}{>}{3}{\end{multicols}}{}}

\usepackage{titlesec}
\usetikzlibrary{
    arrows,shapes,positioning,decorations.markings,
    decorations.pathreplacing,hobby,automata,tikzmark}
\tikzset{
    looped/.style={
        decoration={markings,mark=at position 0.999 with {\arrow[scale=2]{>}}},
        postaction={decorate},
        >=stealth
    },
    straight/.style={
        decoration={markings,mark=at position 1 with {\arrow[scale=2]{>}}},
        postaction={decorate},
        >=stealth
    },
    loopedSF/.style={
        decoration={
            markings,
            mark=at position 0.999 with {\arrow[scale=2]{>}},
            mark=at position 0.5 with {\arrow[scale=2]{>}}},
        postaction={decorate},
        >=stealth
    },
    straightSF/.style={
        decoration={
            markings,
            mark=at position 0.999 with {\arrow[scale=2]{>}},
            mark=at position 0.5 with {\arrow[scale=2]{>}}},
        postaction={decorate},
        >=stealth
    },
    triangle/.style = {fill=white, draw=black, regular polygon, regular polygon sides=3 },
    initial text=$$, % sets the text that appears on the start arrow
    node distance=1.5cm, % specifies the minimum distance between two nodes. Change if necessary.
    ->, % makes the edges directed
}
\titleformat{\subsection}[runin]
{\normalfont\large\bfseries}{\thesubsection}{1em}{}

\titleformat{\subsubsection}[runin]
{\normalfont\normalsize\bfseries}{\thesubsubsection}{1em}{}



\definecolor{mycolor}{rgb}{0.122, 0.435, 0.698}

\newmdenv[innerlinewidth=0.5pt, roundcorner=4pt,linecolor=mycolor,innerleftmargin=6pt,
innerrightmargin=6pt,innertopmargin=6pt,innerbottommargin=6pt]{mybox}

\usepackage[pdf]{graphviz}
\usetikzlibrary{shapes,arrows,calc}
\usepackage[utf8]{inputenc}

\setlength{\droptitle}{-7em}
\posttitle{\par\end{center}}

\newtheorem{definition}{Definition}
\newtheorem{theorem}{Theorem}
\newtheorem{lemma}{Lemma}
\newtheorem{problem}{Problem}
\newtheorem{example}{Example}
\newtheorem{remark}{Remark}

\newcommand{\TODO}[0]{\textcolor{red}{TODO}}
\newcommand{\F}[0]{\mathsf{F}}
\newcommand{\U}[0]{\mathsf{U}}
\newcommand{\G}[0]{\mathsf{G}}
\newcommand{\X}[0]{\mathsf{X}}
\newcommand{\acs}[0]{\textsc{ACL2S}}
\newcommand{\acl}[0]{\textsc{ACL2}}
\newcommand{\pr}[0]{\textsc{Promela}}
\newcommand{\ko}[0]{\textsc{Korg}}
\newcommand{\spn}[0]{\textsc{SPIN}}
\newcommand{\ie}{\emph{i.e.}}
\newcommand{\etal}{\emph{et al.}}

\newcommand{\Pcl}[0]{\textsc{P}}
\newcommand{\NP}[0]{\textsc{NP}}
\newcommand{\SAT}[0]{\texttt{SAT}}
\newcommand{\coNP}[0]{\textsc{coNP}}
\newcommand{\Time}[0]{\text{TIME}}
\newcommand{\Space}[0]{\text{SPACE}}
\newcommand{\threeSAT}[0]{\texttt{3SAT}}

\DeclarePairedDelimiter\abs{\lvert}{\rvert}
\DeclarePairedDelimiter\norm{\lVert}{\rVert}

\usepackage{tikz}
\usetikzlibrary{arrows,chains,matrix,positioning,scopes}

\newlength{\bubblewidth}
\AtBeginDocument{\setlength{\bubblewidth}{.75\textwidth}}
\definecolor{bubblegreen}{RGB}{103,184,104}
\definecolor{bubblegray}{RGB}{241,240,240}

\newcommand{\bubble}[4]{%
  \tcbox[
    colback=#1,
    colframe=#1,
    #2,
  ]{\color{#3}\begin{varwidth}{\bubblewidth}#4\end{varwidth}}%
}

\ExplSyntaxOn
\seq_new:N \l__ooker_bubbles_seq
\tl_new:N \l__ooker_bubbles_first_tl
\tl_new:N \l__ooker_bubbles_last_tl

\NewEnviron{rightbubbles}
 {
  \raggedleft\sffamily
  \seq_set_split:NnV \l__ooker_bubbles_seq { \par } \BODY
  \int_compare:nTF { \seq_count:N \l__ooker_bubbles_seq < 2 }
   {
    \bubble{bubblegreen}{rounded~corners}{white}{\BODY}
   }
   {
    \seq_pop_left:NN \l__ooker_bubbles_seq \l__ooker_bubbles_first_tl
    \seq_pop_right:NN \l__ooker_bubbles_seq \l__ooker_bubbles_last_tl
    \bubble{bubblegreen}{sharp~corners=southeast}{white}{\l__ooker_bubbles_first_tl}\par
    \seq_map_inline:Nn \l__ooker_bubbles_seq
     {
      \bubble{bubblegreen}{sharp~corners=east}{white}{##1}\par
     }
    \bubble{bubblegreen}{sharp~corners=northeast}{white}{\l__ooker_bubbles_last_tl}\par
   }
 }
\NewEnviron{leftbubbles}
 {
  \raggedright\sffamily
  \seq_set_split:NnV \l__ooker_bubbles_seq { \par } \BODY
  \int_compare:nTF { \seq_count:N \l__ooker_bubbles_seq < 2 }
   {
    \bubble{bubblegray}{rounded~corners}{black}{\BODY}
   }
   {
    \seq_pop_left:NN \l__ooker_bubbles_seq \l__ooker_bubbles_first_tl
    \seq_pop_right:NN \l__ooker_bubbles_seq \l__ooker_bubbles_last_tl
    \bubble{bubblegray}{sharp~corners=southwest}{black}{\l__ooker_bubbles_first_tl}\par
    \seq_map_inline:Nn \l__ooker_bubbles_seq
     {
      \bubble{bubblegray}{sharp~corners=west}{black}{##1}\par
     }
    \bubble{bubblegray}{sharp~corners=northwest}{black}{\l__ooker_bubbles_last_tl}\par
   }
 }
\ExplSyntaxOff

\begin{document}
\maketitle

We begin by generalizing our discrepency definition to a multi-party setting.
The general form is basically the same as the 2-party form, except we reason about intersections in cylinders, rather than rectangles in partitions.
\begin{definition}[Multi-Party Discrepency]
Suppose
\[f : \underbrace{\mathbb{B}^n \times ... \times \mathbb{B}^n}_{k \text{ times}} \to \mathbb{B}\]
is a function.  Then the \emph{$k$-party discrepency of $f$} is defined as follows, where $T$ ranges over all cylinder intersections of $f$.
\[\emph{Disc}(f) = \frac{1}{(2^n)^k} \max_T \abs{\sum_{(x_1,...,x_k) \in T} f(x_1,...,x_k)}\]
\end{definition}
For even moderaly sized $n$ and $k$, this definition becomes unreasonable to compute on-the-fly.  But if we could lower-bound the discrepency, for example, using a statistical method, then we could get a slightly looser (but, cheaper) lower-bound on the communication complexity.
First we need two useful but non-intuitive definitions.
\begin{definition}[$(k, n)$-Cube]
A $(k, n)$-cube is a set $D$ of the form
\[D = \{ a_1, a_1' \} \times ... \times \{ a_k, a_k' \}\]
where each $a_i, a_i' \in \mathbb{B}^n$.
To be clear, a point in $d$ is a vector $(x_1, x_2, ..., x_k)$ where each $x_i \in \{ a_i, a_i' \}$.
\end{definition}
\begin{definition}[$\mathcal{E}$]
Let $f : (\mathbb{B}^n)^k \to \mathbb{B}$ be a function.
Then:
\[\mathcal{E}(f) = \underset{\substack{D \text{ is a}\\(k, n)\text{-cube}}}{{\LARGE \text{E}}} \Big[\prod_{\vec{d} \in D} f(\vec{d}) \Big]\]
In other words, $\mathcal{E}(f)$ denotes the expectation over the question,
	\emph{given an arbitrary cube $D$, what is the product of the image of $f$ over all the points $\vec{d} \in D$?}
\end{definition}
Although somewhat scary-looking, this definition pays dividends immediately.
\begin{lemma}[$k$-Party Discrepency Bound]
If $f : (\mathbb{B}^n)^k \to \mathbb{B}$ is a function,
then $\emph{Disc}(f) \leq (\mathcal{E}(f))^{1/2^k}$.
\end{lemma}
Why do we care?  Well, we could approximate $\mathcal{E}(f)$ statistically,
	by checking a ton of random cubes.
	And in this way we could get a decent idea of what an upper bound
	on the discrepency looks like.
	But then recall that the logarithm of the inverse of the discrepency
	lower-bounds the complexity.
	Hence, any upper-bound on discrepency naturally induces a lower-bound on complexity.
	So we've found a way to compute a lower-bound on the complexity without doing
	all the work of computing the discrepency, which is kind of cool.
The proof is pretty complicated, but here's a proof sketch.
\begin{proof}
Given any arbitrary cylinder intersection and $(n, k)$-cube, what is the expectation of the image of points in the cube under $f$?  What if we only consider points falling in the cylinder intersection?
Derive a lower bound on $\mathcal{E}(f)$ which looks something like
\[\mathcal{E}(f) \geq E_{x_1,...,x_k}[f(x_1,....,x_k)(1 \text{ if } (x_1,...,x_k) \in C \text{ else } 0)]^{2k}\]
given a cylinder intersection $C$.
Argue from the definition of the $k$-party discrepency that this gives a natural lower-bound $\mathcal{E}(f) \geq \emph{Disc}(f)^{2k}$.
But this implies \(\emph{Disc}(f) \leq (\mathcal{E}(f))^{1/2^k}\), and we're done.
\end{proof}

% The proof is in the pudding.
% Just kidding, it's an exercise in the book and it looks mega-difficult.
% Instead, here is a super hand-wavey proof sketch.
% \begin{proof}
% By definition, 
% % \[\begin{aligned}
% % \mathcal{E}(f) 
% % 	& = \underset{\substack{D \text{ is a}\\(k, n)\text{-cube}}}{{\LARGE \text{E}}} \Big[\prod_{\vec{d} \in D} f(\vec{d}) \Big] \\
% % 	& = \underset{
% % 		D \in (\mathbb{B}^n \times \mathbb{B}^n)^k
% % 	}{{\LARGE \text{E}}} \Big[\prod_{\vec{d} \in D} f(\vec{d}) \Big] \\
% % 	& = \underset{D_1 \in \mathbb{B}^n \times \mathbb{B}^n}{\LARGE \text{E}} ...
% % 	    \underset{D_k \in \mathbb{B}^n \times \mathbb{B}^n}{\LARGE \text{E}}
% % 	    \Big[ \prod_{x_1 \in \{ d_1, d_1' \}}
% % 	    	  ...
% % 	    	  \prod_{x_{k-1} \in \{ d_{k-1}, d_{k-1}' \}}
% % 	    	  \prod_{x_k \in \{ d_k, d_k' \}}
% % 	    	  f(x_1, ..., x_{k-1} x_k)
% % 	    \Big] \\
% % 	 & = \underset{\{ d_1, d_1' \} \in \mathbb{B}^n \times \mathbb{B}^n}{\LARGE \text{E}} ...
% % 	    \underset{\{ d_k, d_k' \} \in \mathbb{B}^n \times \mathbb{B}^n}{\LARGE \text{E}}
% % 	    \Big[ \prod_{x_1 \in \{ d_1, d_1' \}}
% % 	    	  ...
% % 	    	  \prod_{x_{k-1} \in \{ d_{k-1}, d_{k-1}' \}}
% % 	    	  f(x_1, ..., x_{k-1}, d_k) * f(x_1, ..., x_{k-1}, d_k')
% % 	    \Big] \\
% % 	& = \underset{\{ d_1, d_1' \} \in \mathbb{B}^n \times \mathbb{B}^n}{\LARGE \text{E}} ...
% % 	    \underset{\{ d_k, d_k' \} \in \mathbb{B}^n \times \mathbb{B}^n}{\LARGE \text{E}}
% % 	    \Big[ \prod_{x_1 \in \{ d_1, d_1' \}}
% % 	    	  ...
% % 	    	  \prod_{x_{k-1} \in \{ d_{k-1}, d_{k-1}' \}}
% % 	    	  f(x_1, ..., x_{k-1}, d_k) * f(x_1, ..., x_{k-1}, d_k')
% % 	    \Big] \\
% % 	& = \underset{\{ d_1, d_1' \} \in \mathbb{B}^n \times \mathbb{B}^n}{\LARGE \text{E}} ...
% % 	    \underset{\{ d_k, d_k' \} \in \mathbb{B}^n \times \mathbb{B}^n}{\LARGE \text{E}}
% % 	    \Big[ \prod_{x_1 \in \{ d_1, d_1' \}}
% % 	    	  ...
% % 	    	  \prod_{x_{k-1} \in \{ d_{k-1}, d_{k-1}' \}}
% % 	    	  f(x_1, ..., x_{k-1}, d_k) * f(x_1, ..., x_{k-1}, d_k')
% % 	    \Big] \\
% % 	& = \underset{\{ d_1, d_1' \} \in \mathbb{B}^n \times \mathbb{B}^n}{\LARGE \text{E}} ...
% % 	    \underset{\{ d_{k-1}, d_{k-1}' \} \in \mathbb{B}^n \times \mathbb{B}^n}{\LARGE \text{E}}
% % 	    \Big[ \prod_{x_1 \in \{ d_1, d_1' \}}
% % 	    	  ...
% % 	    	  \prod_{x_{k-1} \in \{ d_{k-1}, d_{k-1}' \}}
% % 	    	  \underset{d_k \in \mathbb{B}^n}{\LARGE \text{E}}
% % 	    	  f(x_1, ..., x_{k-1}, d_k) 
% % 	    	  \underset{d_k' \in \mathbb{B}^n}{\LARGE \text{E}}
% % 	    	  f(x_1, ..., x_{k-1}, d_k')
% % 	    \Big] \\
% % 	& = \underset{\{ d_1, d_1' \} \in \mathbb{B}^n \times \mathbb{B}^n}{\LARGE \text{E}} ...
% % 	    \underset{\{ d_{k-1}, d_{k-1}' \} \in \mathbb{B}^n \times \mathbb{B}^n}{\LARGE \text{E}}
% % 	    \Big[ \prod_{x_1 \in \{ d_1, d_1' \}}
% % 	    	  ...
% % 	    	  \prod_{x_{k-1} \in \{ d_{k-1}, d_{k-1}' \}}
% % 	    	  (\underset{d_k \in \mathbb{B}^n}{\LARGE \text{E}}
% % 	    	  f(x_1, ..., x_{k-1}, d_k))^2
% % 	    \Big] \\
% % 	& = (\underset{d_1 \in \mathbb{B}^n}{\LARGE \text{E}} [
% % 	      (\underset{d_2 \in \mathbb{B}^n}{\LARGE \text{E}} [
% % 	      	...
% % 	      		(\underset{d_k \in \mathbb{B}^n}{\LARGE \text{E}} [ f(d_1,...,d_k) ])^2 ])^2 .... ])^2 \\
% % 	& \geq \Big( 
% % 	\underset{d_1 \in \mathbb{B}^n}{\LARGE \text{E}} [
% % 		\underset{d_2 \in \mathbb{B}^n}{\LARGE \text{E}} [
% % 			...
% % 			\underset{d_k \in \mathbb{B}^n}{\LARGE \text{E}} [
% % 				f(d_1,...,d_k) ] ... ]]
% % 	\Big)^{2k}
% % \end{aligned}\]
% The next part of the proof is to relate the term on the right-hand-side to the discrepency.  The trick is to show that the expectation of the function is at most the di
% \end{proof}
\end{document}