\title{The 2-Party Discrepency Method}
\author{Max von Hippel}
\date{\today}

\documentclass[12pt]{article}

\usepackage{usenixy}
\usepackage{
  amsmath,mathtools,amssymb,fullpage,
  amsthm,centernot,float,diagbox,adjustbox,
  graphicx,caption,subcaption,listings,
  enumerate,xcolor,hyperref,color,soul,
  stmaryrd,amsrefs,titling,tikz,
  mathrsfs,etoolbox,refcount,multicol,MnSymbol,
  yfonts,framed,relsize,colortbl,pgf-umlsd,varwidth,
  environ,xparse}
\lstset{
  language=Lisp,
  basicstyle=\ttfamily,
}
\usepackage[many]{tcolorbox}


\usepackage[framemethod=tikz]{mdframed}

\newcommand{\grayleftbar}[1]{\textcolor[gray]{0.7}{%
     \begin{leftbar}\textcolor{black}{#1}\end{leftbar}}}   
\newcommand{\darkleftbar}[1]{\textcolor[gray]{0.3}{%
     \begin{leftbar}\textcolor{black}{#1}\end{leftbar}}}   

\newcounter{countitems}
\newcounter{nextitemizecount}
\newcommand{\setupcountitems}{%
  \stepcounter{nextitemizecount}%
  \setcounter{countitems}{0}%
  \preto\item{\stepcounter{countitems}}%
}
\makeatletter
\newcommand{\computecountitems}{%
  \edef\@currentlabel{\number\c@countitems}%
  \label{countitems@\number\numexpr\value{nextitemizecount}-1\relax}%
}
\newcommand{\nextitemizecount}{%
  \getrefnumber{countitems@\number\c@nextitemizecount}%
}
\newcommand{\previtemizecount}{%
  \getrefnumber{countitems@\number\numexpr\value{nextitemizecount}-1\relax}%
}
\makeatother    
\newenvironment{AutoMultiColItemize}{%
\ifnumcomp{\nextitemizecount}{>}{3}{\begin{multicols}{2}}{}%
\setupcountitems\begin{itemize}}%
{\end{itemize}%
\unskip\computecountitems\ifnumcomp{\previtemizecount}{>}{3}{\end{multicols}}{}}

\usepackage{titlesec}
\usetikzlibrary{
    arrows,shapes,positioning,decorations.markings,
    decorations.pathreplacing,hobby,automata}
\tikzset{
    looped/.style={
        decoration={markings,mark=at position 0.999 with {\arrow[scale=2]{>}}},
        postaction={decorate},
        >=stealth
    },
    straight/.style={
        decoration={markings,mark=at position 1 with {\arrow[scale=2]{>}}},
        postaction={decorate},
        >=stealth
    },
    loopedSF/.style={
        decoration={
            markings,
            mark=at position 0.999 with {\arrow[scale=2]{>}},
            mark=at position 0.5 with {\arrow[scale=2]{>}}},
        postaction={decorate},
        >=stealth
    },
    straightSF/.style={
        decoration={
            markings,
            mark=at position 0.999 with {\arrow[scale=2]{>}},
            mark=at position 0.5 with {\arrow[scale=2]{>}}},
        postaction={decorate},
        >=stealth
    },
    triangle/.style = {fill=white, draw=black, regular polygon, regular polygon sides=3 },
    initial text=$$, % sets the text that appears on the start arrow
    node distance=1.5cm, % specifies the minimum distance between two nodes. Change if necessary.
    ->, % makes the edges directed
}
\titleformat{\subsection}[runin]
{\normalfont\large\bfseries}{\thesubsection}{1em}{}

\titleformat{\subsubsection}[runin]
{\normalfont\normalsize\bfseries}{\thesubsubsection}{1em}{}



\definecolor{mycolor}{rgb}{0.122, 0.435, 0.698}

\newmdenv[innerlinewidth=0.5pt, roundcorner=4pt,linecolor=mycolor,innerleftmargin=6pt,
innerrightmargin=6pt,innertopmargin=6pt,innerbottommargin=6pt]{mybox}

\usepackage[pdf]{graphviz}
\usetikzlibrary{shapes,arrows}
\usepackage[utf8]{inputenc}

\setlength{\droptitle}{-7em}
\posttitle{\par\end{center}}

\newtheorem{definition}{Definition}
\newtheorem{theorem}{Theorem}
\newtheorem{lemma}{Lemma}
\newtheorem{problem}{Problem}
\newtheorem{example}{Example}
\newtheorem{remark}{Remark}

\newcommand{\TODO}[0]{\textcolor{red}{TODO}}
\newcommand{\F}[0]{\mathsf{F}}
\newcommand{\U}[0]{\mathsf{U}}
\newcommand{\G}[0]{\mathsf{G}}
\newcommand{\X}[0]{\mathsf{X}}
\newcommand{\acs}[0]{\textsc{ACL2S}}
\newcommand{\acl}[0]{\textsc{ACL2}}
\newcommand{\pr}[0]{\textsc{Promela}}
\newcommand{\ko}[0]{\textsc{Korg}}
\newcommand{\spn}[0]{\textsc{SPIN}}
\newcommand{\ie}{\emph{i.e.}}
\newcommand{\etal}{\emph{et al.}}

\newcommand{\Pcl}[0]{\textsc{P}}
\newcommand{\NP}[0]{\textsc{NP}}
\newcommand{\SAT}[0]{\texttt{SAT}}
\newcommand{\coNP}[0]{\textsc{coNP}}
\newcommand{\Time}[0]{\text{TIME}}
\newcommand{\Space}[0]{\text{SPACE}}
\newcommand{\threeSAT}[0]{\texttt{3SAT}}

\DeclarePairedDelimiter\abs{\lvert}{\rvert}
\DeclarePairedDelimiter\norm{\lVert}{\rVert}

\usepackage{tikz}
\usetikzlibrary{arrows,chains,matrix,positioning,scopes}

\newlength{\bubblewidth}
\AtBeginDocument{\setlength{\bubblewidth}{.75\textwidth}}
\definecolor{bubblegreen}{RGB}{103,184,104}
\definecolor{bubblegray}{RGB}{241,240,240}

\newcommand{\bubble}[4]{%
  \tcbox[
    colback=#1,
    colframe=#1,
    #2,
  ]{\color{#3}\begin{varwidth}{\bubblewidth}#4\end{varwidth}}%
}

\ExplSyntaxOn
\seq_new:N \l__ooker_bubbles_seq
\tl_new:N \l__ooker_bubbles_first_tl
\tl_new:N \l__ooker_bubbles_last_tl

\NewEnviron{rightbubbles}
 {
  \raggedleft\sffamily
  \seq_set_split:NnV \l__ooker_bubbles_seq { \par } \BODY
  \int_compare:nTF { \seq_count:N \l__ooker_bubbles_seq < 2 }
   {
    \bubble{bubblegreen}{rounded~corners}{white}{\BODY}
   }
   {
    \seq_pop_left:NN \l__ooker_bubbles_seq \l__ooker_bubbles_first_tl
    \seq_pop_right:NN \l__ooker_bubbles_seq \l__ooker_bubbles_last_tl
    \bubble{bubblegreen}{sharp~corners=southeast}{white}{\l__ooker_bubbles_first_tl}\par
    \seq_map_inline:Nn \l__ooker_bubbles_seq
     {
      \bubble{bubblegreen}{sharp~corners=east}{white}{##1}\par
     }
    \bubble{bubblegreen}{sharp~corners=northeast}{white}{\l__ooker_bubbles_last_tl}\par
   }
 }
\NewEnviron{leftbubbles}
 {
  \raggedright\sffamily
  \seq_set_split:NnV \l__ooker_bubbles_seq { \par } \BODY
  \int_compare:nTF { \seq_count:N \l__ooker_bubbles_seq < 2 }
   {
    \bubble{bubblegray}{rounded~corners}{black}{\BODY}
   }
   {
    \seq_pop_left:NN \l__ooker_bubbles_seq \l__ooker_bubbles_first_tl
    \seq_pop_right:NN \l__ooker_bubbles_seq \l__ooker_bubbles_last_tl
    \bubble{bubblegray}{sharp~corners=southwest}{black}{\l__ooker_bubbles_first_tl}\par
    \seq_map_inline:Nn \l__ooker_bubbles_seq
     {
      \bubble{bubblegray}{sharp~corners=west}{black}{##1}\par
     }
    \bubble{bubblegray}{sharp~corners=northwest}{black}{\l__ooker_bubbles_last_tl}\par
   }
 }
\ExplSyntaxOff

\begin{document}
\maketitle

\section{Introduction}

When we partition $M(f)$ into some number of rectangles,
	the sizes of the rectangles must add up to the size of $M(f)$.
Hence, if $\chi(f) \leq K$ for some integer $K$,
	then $M(f)$ must have a monochromatic rectangle containing at least $2^n * 2^n / K$ entries.
\begin{proof}
Suppose that $\chi(f) \leq K$ for some integer $K$.
If $\chi(f) = K$ then there exists a partioning of $M(f)$ into $K$ monochromatic rectangles, in which case at least one of those rectangles must have size $\geq \abs{M(f)} / K$, i.e., $2^n * 2^n / K$.  On the other hand if $\chi(f) < K$ then $\chi(f) = K'$ for some $K' < K$ and then $M(f)$ can be partitioned into $K'$ monochromatic rectangles, at least of which has size $\geq \abs{M(f)} / K'$, which is strictly larger than $\abs{M(f)} / K$.  So in either case the result holds and we are done.
\end{proof}
Now, suppose that $M(f)$ contains a monochromatic rectangle $A \times B$ having at least $2^n * 2^n / K$ entries.
Since $A \times B$ is monochromatic, this implies that:
\[\sum_{a \in A, b \in B} (-1)^{M_{a,b}} = \begin{cases}
-1 * \text{ the size of the rectangle } A \times B & \text{ if it's colored } 1 \\
 1 * \text{ the size of the rectangle } A \times B & \text{ if it's colored } 0\end{cases}\]
So if we wrap an absolute value above our sum, we get:
\[\abs{\sum_{a \in A, b \in B} (-1)^{M_{a,b}}} = \text{ the size of the rectangle } A \times B\]
But we already assumed that $A \times B$ has at least $2^n * 2^n / K$ entries, hence:
\[\abs{\sum_{a \in A, b \in B} (-1)^{M_{a,b}}} = \text{ the size of the rectangle } A \times B \geq 2^n * 2^n / K\]
Let's divide both size by $2^n * 2^n$, for fun and profit.
\[\frac{1}{2^n * 2^n}\abs{\sum_{a \in A, b \in B} (-1)^{M_{a,b}}} \geq 1 / K\]
We are mathematicians, and mathematicians like to name things.
Let's do that.
\begin{definition}[Discrepency]
The \emph{discrepency} of a rectangle $A \times B$ of $M(f)$ is exactly the following.
\[\emph{Disc}(A \times B) = \frac{1}{2^n * 2^n}\abs{\sum_{a \in A, b \in B} (-1)^{M_{a,b}}}\]
The \emph{discrency of $M(f)$} is the maximum discrepency among all its rectangles.
\end{definition}
Now that we've named this thing, let's re-write our inequality.
\[\emph{Disc}(A \times B) \geq 1 / K\]
Taking inverses:
\[\frac{1}{\emph{Disc}(A \times B)} \leq K\]
Certainly $\chi(f) \leq \chi(f)$, so supplanting $\chi(f)$ for $K$ in the statement, we get:
\begin{equation}
\frac{1}{\emph{Disc}(A \times B)} \leq \chi(f)
\label{eqRes}
\end{equation}
This result generalizes as follows.
\begin{lemma}[2-Party Discrepency Method]
Suppose $f : \mathbb{B}^n \times \mathbb{B}^n \to \mathbb{B}$ is a function.
Then Equation~\ref{eqRes} holds.
\end{lemma}
\end{document}